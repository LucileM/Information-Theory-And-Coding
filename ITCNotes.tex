\documentclass[10pt,a4paper,landscape]{article}
\usepackage{multicol}
\usepackage{calc}
\usepackage{ifthen}
\usepackage[landscape]{geometry}
\usepackage{amsmath,amsthm,amsfonts,amssymb,mathtools}
\usepackage{color,graphicx}
\usepackage[dvipsnames]{xcolor}
\usepackage{hyperref}
\usepackage{listings}
\usepackage{underscore}
\usepackage{todonotes}

% style
% Cheatsheet style

% This sets page margins to .5 inch if using letter paper, and to 1cm
% if using A4 paper. (This probably isn't strictly necessary.)
% If using another size paper, use default 1cm margins.
\ifthenelse{\lengthtest{\paperwidth = 11in}}
  % Then
  { \geometry{top=.5in,left=.5in,right=.5in,bottom=.5in} }
  % Else
  { \ifthenelse{\lengthtest{\paperwidth = 297mm}}
    {\geometry{top=1cm,left=1cm,right=1cm,bottom=1cm} }
    {\geometry{top=1cm,left=1cm,right=1cm,bottom=1cm} }
  }

% Turn off header and footer
\pagestyle{empty}

% Redefine section commands to use less space
\makeatletter
\renewcommand{\section}{\@startsection{section}{1}{0mm}%
                                {-1ex plus -.5ex minus -.2ex}%
                                {0.5ex plus .2ex}%x
                                {\color{darkdarkred}\normalfont\large\bfseries}}
\renewcommand{\subsection}{\@startsection{subsection}{2}{0mm}%
                                {-1explus -.5ex minus -.2ex}%
                                {0.5ex plus .2ex}%
                                {\color{darkdarkred}\normalfont\normalsize\bfseries}}
\renewcommand{\subsubsection}{\@startsection{subsubsection}{3}{0mm}%
                                {-1ex plus -.5ex minus -.2ex}%
                                {1ex plus .2ex}%
                                {\normalfont\small\bfseries}}
\makeatother

% Define BibTeX command
\def\BibTeX{{\rm B\kern-.05em{\sc i\kern-.025em b}\kern-.08em
    T\kern-.1667em\lower.7ex\hbox{E}\kern-.125emX}}

% Don't print section numbers
\setcounter{secnumdepth}{0}

\setlength{\parindent}{0pt}
\setlength{\parskip}{0pt plus 0.5ex}

% Setting colors
\definecolor{lightgray}{rgb}{0.7,0.7,0.7}
\definecolor{lightergray}{rgb}{0.9,0.9,0.9}
\definecolor{darkblue}{rgb}{0.4,0.4,1}
\definecolor{darkred}{rgb}{0.9,0.2,0.2}
\definecolor{darkdarkred}{rgb}{0.6,0.0,0.0}
\definecolor{lightred}{rgb}{1,0.6,0.6}
\definecolor{lightgreen}{rgb}{0.6,1,0.6}
\definecolor{lightblue}{rgb}{0.6,0.8,1}
\definecolor{darkgreen}{rgb}{0.4,1,0.4}

% Set code listing style
\lstset {
    backgroundcolor=\color{lightgray},
    basicstyle=\ttfamily\scriptsize,
    breaklines=true,
}

\lstdefinestyle{bb}{
    backgroundcolor=\color{lightergray},
    frame=L,
    xleftmargin=\parindent,
}

% Remove `itemize` indentation
\usepackage{enumitem}
\setlist[itemize]{leftmargin=*}

% Set hyperlink style
\hypersetup{hidelinks}

% Enable figures
\newenvironment{colfig}
  {\par\medskip\noindent\minipage{\linewidth}}
  {\endminipage\par\medskip}

% Enable arg min/max math operators
\DeclareMathOperator*{\argmin}{arg\,min}
\DeclareMathOperator*{\argmax}{arg\,max}


% Shorthands
\renewcommand{\bf}[1]{\ensuremath{\mathbf{#1}}}
\newcommand{\E}{\mathrm{E}}
\newcommand{\Var}{\mathrm{Var}}
\newcommand{\Cov}{\mathrm{Cov}}
\newcommand{\balpha}{\boldsymbol\alpha}
\newcommand{\bbeta}{\boldsymbol\beta}
\newcommand{\bdelta}{\boldsymbol\delta}
\newcommand{\btheta}{\boldsymbol\theta}
\newcommand{\bPhi}{\boldsymbol\Phi}
\newcommand{\concept}[1]{\textcolor{RoyalBlue}{#1}} 
\newcommand{\subconcept}[1]{\textcolor{PineGreen}{\textit{#1}}}

% Allows to easily edit out parts that are dispensible
\newcommand{\opt}[1]{#1}
% Modifications in order to maximize information density
\ifdefined \longversion % Nothing
\else
  % Hide optional content
  \renewcommand{\opt}[1]{}
  % Make titles smaller
  \renewcommand{\section}[1]{
    \vspace{-0.3cm}
    \begin{center}
      \color{Bittersweet}
      \hrulefill{\small~~#1~~}\hrulefill
    \end{center}
    \vspace{-0.3cm}
  }
  \renewcommand{\subsection}[1]{\section{#1}}
\fi

\pdfinfo{
  /Title (Information Theory and Coding)
  /Creator (Lucile Madoulaud)
  /Author (Lucile Madoulaud)
  /Subject (Information Theory and Coding)
  /Keywords (ITC, information, communication, theory, coding)
}

% -----------------------------------------------------------------------

\begin{document}
\title{Information Theory and Coding}

\raggedright
\footnotesize
\sffamily
\begin{multicols*}{4}

% multicol parameters
% These lengths are set only within the two main columns
%\setlength{\columnseprule}{0.25pt}
\setlength{\premulticols}{1pt}
\setlength{\postmulticols}{1pt}
\setlength{\multicolsep}{1pt}
\setlength{\columnsep}{2pt}

\opt {
  \begin{center}
  \Large{Information Theory and Coding}
  \end{center}
}

% ----------
\section{Source Coding}

\concept{Introduction}
Diagram of a general communication system.
\subconcept{Discrete sources} output of the source is in discrete time and discrete valued.
\subconcept{Source Coding} representation of information sources in bits.
\subconcept{Source Code Function} $C:U\mapsto\{0,1\}^*=\{\emptyset,0,1,00,...\}$.

\concept{Non-Singular Codes} 
A code $C$ is \subconcept{singular} if $\exists u \neq v /\ C(u)=C(v)$. A code $C$ is \subconcept{non-singular} if it is not singular.
With a code $C$ define for a positive integer $n$ :
$C^n:U^n\mapsto\{0,1\}^*$ as $C^n(u_1,u_2,...,u_n)=C(u_1)C(u_2)...C(u_n)$
$C^*:U^*\mapsto\{0,1\}^*$ as $C^*(u_1u_2...u_n)=C(u_1)C(u_2)...C(u_n)$

\concept{Uniquely Decodable Codes}
A code $C$ is said to be \subconcept{uniquals decodable} if $C^*$ is non-singular.We want our codes to be uniquals decodable.

\concept{Prefix-Free Codes} 
A sequence $u_1,... u_n$ is a \subconcept{prefix} of $v_1,...,v_n$ if $n \geqslant m /\ u_1=v_1,...,u_m=v_m$.
A code $C$ is said to be \subconcept{prefix-free} if $\forall u \neq v$ $C(u)$ is not a prefix of $C(v)$.

\subconcept{Theorem} A prefix-free code is uniquely decodable.
(In a binary-tree representation of a PF code all codewords are found on the leaves).

\concept{Kraft's Inequality for PF Codes} 
\subconcept{Theorem}
If $C$ is PF then 
$Kraftsum(C) \triangleq \sum_{u \in U }2^{-length(C(u))} \le 1$.
\subconcept{Proposition} $Kraftsum(C^n)=\left[ Kraftsum(C) \right]^n$.

\concept{Kraft's Inequality for extensions of codes} 
\subconcept{Proposition} Suppose $C:U\mapsto\{0,1\}^*$ is a non-singular code then $Kraftsum(C) = \sum_{u \in U }2^{-length(C(u))} \le 1 + max\left[ length(C(u)) \right]$

\concept{Kraft's Inequality for uniquely decodable codes}
\subconcept{Theorem} If $C$ is a uniquely decodable code then $Kraftsum(C) \le 1$.
\subconcept{Corollary} If $C$ is a uniquely decodable code then there exists a PF code $C'$ such that $length(C(u)) = length(C'(u))$.

\concept{Reverse Kraft's inequality}
\subconcept{Theorem} Given an alphabet $U$ and a function $l:u \mapsto \{0,1,2,3,...\} /\ \sum_{u \in U }2^{-length(C(u))} \le 1$ then there exist a PF code $C:U\mapsto\{0,1\}^* /\ \forall u \in U$ $length(C(u))=l(u) $ 


\concept{Sources} A source producer a sequence $u_1,u_2,u_3,...$ each $u_i \in U$ being random variables.
A \subconcept{memory-less} source is one where $u_1,u_2,...$ are independent.
A \subconcept{stationary} source is one where each $(u_i,...,u_{i+n-1})$ has the same statistics as $(u_1,...,u_n)$ for each $i$ and each $n$.
A memory-less and stationary source is equivalent to $u_1,u_2,...$ are \subconcept{independent, identically distributed (iid)}.

\concept{Expected codeword length}
$E\left[ length(C(u)) \right]$ average number of bits/letter the code uses to represent the source. We want to minimize it and $C$ to be uniquely decodable.

% ----------

\section{Entropy}
\subconcept{Lemma} $ln(z)\le z^{-1}$ with eq if $z=1$.
\subconcept{Property} $0 \le H(U) \le log |U|$
\concept{Entropy as a lower-bound to the expected codeword length} 
\subconcept{Theorem} For any uniquely decodable code $C$ for a source $U$, we have $E\left[ length(C(u)) \right] \ge \sum_{u}p(u)log_2 \frac{1}{p(u)} \triangleq H(u)$

\concept{Existence of PF codes with average length at most entropy + 1}
\subconcept{Theorem} Given source $U$ there exists a PF code $C$ s.t. 
$E\left[ length(C(u)) \right] \ge H(u)+1$
 
\concept{Entropy of multiple random variables} 
\subconcept{Property} Suppose $U$ and $V$ are ind. RV.
Then $H(UV)=H(U)+H(V)$.
\subconcept{Observe} Suppose we have $U_1U_2...$ iid. If we use a code $C$ to represent $n$ letters at time., we will have $H(U_1...U_n) \le E\left[ length(C(U_1...U_n)) \right] \le H(U_1...U_n)+1$.
\subconcept{Also} $\frac{1}{n} H(U_1...U_n) = H(U_1)$ (iid of U).

\concept{Properties of optimal codes} 
\subconcept{1} If $p(u) < p(v)$ then $l(u) \ge l(v)$.
\subconcept{2} In an optimal PF code there are more than 2 longest codewords. If not the longest codeword can be shortered without violating the PF condition.
\subconcept{3} Among optimal codes, there is one for the two least probable symbols are siblings.

\concept{Huffman procedure} 
Procedure to design the optimal code.
\subconcept{1} Given prob $p_1,p_2,...,p_{k-1},p_k$. Start with the two smallest prob.
\subconcept{2} Group them together as the binary descendant of a node.
\subconcept{3} Repeat until one node is left.

\concept{Equivalence of PF codes and strategy for guessing via binary questions} TODO

\concept{Interpretation of entropy as expected number of questions for guessing the random variable} TODO

% --------
\section{Mutual Information}

\concept{Conditional Entropy and Mutual Information} 
\subconcept{Conditional Entropy} $H(U|V=v)=\sum_{u}p(u|v)log \frac{1}{p(u|v)}$
$H(U|V)=\sum_{v}p(v)H(U|V=v)$.
\subconcept{Conjecture} $H(U|V) \le H(U)$.
\subconcept{Mutual Information} $I(U;V)=H(U)+H(V)-H(UV)$ is the saving in the number of questions to given $U$ by the knowledge of $V$.
\subconcept{Lemma} Suppose $W$ is an alphabet and $p$ and $q$ are two prob distribution in W. Then, $\sum_{w}p(w)log \frac{p(w)}{q(w)} \ge 0$ with eq. iff $p=q$.
\subconcept{Theorem} $I(U;V)\ge 0$ with eq iff $U$ and $V$ are independent.
\subconcept{Conditional Mutual Information} 
$I(U;V|W)=H(U|W)+H(V|W)-H(UV|W)=H(U|W)-H(U|VW)=H(V|W)-H(V|UW)$
\subconcept{Theorem} $I(U;V|W)\ge 0$ with eq. iff $U$,$V$ are independent conditional in $W$ $\equiv U - V - W$.
% find the markov chain symbol

\concept{Chain Rules for entropy and mutual information} 
\subconcept{Theorem} $H(UV)=H(U)+H(V|U)=H(V)+H(U|V)$
\subconcept{Theorem} $H(U_1...U_n)=H(U_1)+H(U_2|U_1)+...+H(U_n|U_1...U_{n-1})$
\subconcept{Theorem} $I(U_1...U_n,V)=I(U_1;V)+...+I(U_n;V|U_1...U_{n-1})$
 
\concept{Review of Markov Chain} 
Suppose $X$,$Y$,$Z$ are RVs. We can write $p(xyz)=p(x)p(y|x)p(z|xy)$.
Because $p(y|x)=\frac{p(xy)}{p(x)}$.
If $X-Y-Z$ then $p(xyz)=p(x)p(y|x)p(z|y)$. 
Suppose $U_1-U_2-...-U_n$ then $H(U_1...U_n)=H(U_1)+H(U_2|U_1)+...+H(U_n|U_{n-1})$

\concept{Data Processing Inequality} 
\subconcept{Theorem} Suppose $U-V-W$ then $I(U;W) \le I(U;V)$
\subconcept{Corollary} If $U-V-W$ then $I(U;W) \le I(V;W)$.
\subconcept{Corollary} If $U-V-W-X$ then $I(U;X) \le I(V;W)$.
$I(UV;W)=I(U;W)+I(V;W|U)=I(V;W)+I(U;W|V)$.

\concept{Entropy Rate}
Given a stochastic process $U_1,U_2...$ we define its \subconcept{entropy rate}
$H(U)=\lim_{n \to \infty} \frac{1}{n}H(u_1...u_n)$ if the limit exists.

\concept{Entropy Rate of Stationary Processes} 
\subconcept{Theorem} If $u_1,u_2,...$ is stationary process, then the entropy rate exists and $\lim_{n \to \infty} \frac{1}{n}H(u_1...u_n) = \lim_{n \to \infty}H(u_n|u_1...u_{n-1})$

\concept{Coding Theorem for Stationary Sources} 
\subconcept{Theorem} If 

\concept{Fixed-to-Fixed Length Source Codes} TODO

% ------------
\section{Typicality}

\concept{Typicality} TODO

\concept{Properties of Typical Sets} TODO

\concept{Asymptotic Equipartition Property} TODO

\concept{Source Coding by AEP} TODO

\concept{Variable-to-Fixed Length Source Codes} TODO

\concept{Valid and Prefix-Free Dictionaries} TODO

\concept{Relationship between word- and letter-entropies for valid, prefix-free dictionaries} TODO

% --------------
\section{Tunstall procedure}

\concept{Tunstall procedure} TODO 

\concept{Analysis of Tunstall procedure} TODO

\concept{Universal Source Coding} TODO

\concept{Lempel–Ziv method} TODO 

\concept{Analysis of Lempel–Ziv} TODO

\concept{Information-Lossless FSM Compressors} TODO

\concept{Lower bound on the output length of an IL FSM Compressor} TODO

\concept{LZ Compressibility of sequences} TODO

\concept{Optimality of Lempel–Ziv} TODO

% -----------
\section{Channels}

\concept{Communication Channels} TODO

\concept{Discrete Memoryless Channels} TODO

\concept{Examples of Discrete Memoryless Channels (BSC and BEC)} TODO

\concept{Transmission with or without feedback} TODO

\concept{Channel Capacity} TODO

\concept{Fano's Inequality} TODO

\concept{Converse to the Channel Coding Theorem	} TODO
			
\concept{Proof of the Channel Coding Theorem} TODO

\concept{Capacity of BSC and BEC } TODO

\concept{Jensen's Inequality } TODO

\concept{Concavity of Mutual Information in Input Distribution} TODO
 
\concept{KKT Conditions} TODO

\concept{KKT Conditions (cont'd)} TODO
 
\concept{Application of KKT: Capacity of Z Channel} TODO
 
\concept{Continuous Alphabet: Differential Entropy} TODO

\concept{Properties of differential entropy} TODO
 
\concept{Entropy-typical seqeuences} TODO
 
\concept{Quantization} TODO
 
\concept{Entropy of Gaussian distribution} TODO

\concept{Capacity under cost constraint} TODO
 
\concept{Capacity of AWGN} TODO
 
\concept{Converse to the channel coding theorem with cost constraint} TODO
 
\concept{Parallel Gaussian channels (water-filling)} TODO

\concept{Proof of Channel Coding Theorem for general channels via Threshold Decoding} TODO

\concept{Channel Codes} TODO
 
\concept{Minimum Distance} TODO
 
\concept{Singleton Bound} TODO
 
\concept{Sphere-packing Bound} TODO
 
\concept{Gilbert–Varshamov Bound} TODO

\concept{Linear Codes} TODO
 
\concept{Generator Matrix} TODO
 
\concept{Parity-check Matrix} TODO
 
\concept{Hamming Codes} TODO

\concept{Reed–Solomon Codes} TODO

\concept{Polar Codes} TODO

% ---------- Page break
\ifdefined \longversion % Nothing
\else \newpage
\fi



% ---------- Credits
\section{Credits}
Most content taken from the lecture notes of Emre Telatar's \href{http://ipg.epfl.ch/doku.php?id=en:courses:2015-2016:itc}{Information Theory and Coding class} at EPFL, 2015.

% ---------- Footer
\vspace{0.5cm}
\hrule
\vspace{0.5cm}
\tiny
Rendered \today. Written by Lucile Madoulaud.

\end{multicols*}
\end{document}
